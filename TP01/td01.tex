\documentclass{article}
\usepackage[utf8]{inputenc}
\usepackage{algpseudocode}
\usepackage{algorithm}

\title{Algorithmique Avancée TD01}
\author{Hugo Demaret}
\date{September 2021}
\begin{document}
\maketitle
\section*{Exercice A}
\section*{Exercice B}
Voir sur :\\
https://github.com/HugoDemaret/linked\_list\\
https://github.com/HugoDemaret/doublelinked\_list\\
https://github.com/HugoDemaret/binary\_tree\\

\section*{Exercice C}
\subsection*{1 -}
\begin{algorithmic}
\State $i \gets 10$
\end{algorithmic}
\subsection*{2 -}
\textit{Soit T un graphe avec n sommets. Démontrer que les propriétés suivantes sont équivalentes.}\\\\
\textbf{(1) T est un arbre}\\
\textit{La définition d'un arbre est : Graphe acyclique et connexe.}\\\\
\textbf{(2) T est un graphe connexe et acyclique}\\
\textit{C'est en fait l'une des définitions d'un arbre.}\\\\
\textbf{(3) T est un graphe connexe avec n-1 arrête}\\
\textit{Un graphe cyclique à n sommets possède au minimum n arrêtes. Donc un graphe connexe à n-1 arrêtes est acyclique. C'est un arbre.}\\\\
\textbf{(4) T est un graphe acyclique avec n-1 arrête}\\
\textit{T est un graphe simple. Un graphe acyclique est un graphe simple (la boucle serait un cycle). Le graphe possède n-1 arrêtes, donc chaque sommet possède au moins une arrête (parfois en commun avec un autre sommet). Donc le graphe est connexe. C'est donc un arbre.}\\\\
\textit{Les propriétés précédentes décrivant toutes un arbre, elles sont équivalentes.}\\\\




\end{document}